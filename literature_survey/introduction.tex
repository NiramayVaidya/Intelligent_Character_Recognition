\documentclass[main.tex]{subfiles}
\begin{document}

\begin{flushleft}
\Large{\bf{Introduction}}
\end{flushleft}
\vspace{1.5mm}
\justify
Machine simulation of human functions has been a very challenging research
field since the advent of digital computers. In some areas, which require
certain amounts of intelligence, such as number crunching or chess playing,
tremendous improvements are achieved. On the other hand, humans still outperform
even the most powerful computers in the relatively routine functions such as
vision. Machine simulation of human reading is one of these areas, which has
been the subject of intensive research for the last three decades, yet it is
still far from the final frontier.
\par In general, handwriting recognition is classified into two types as offline
and online handwriting recognition methods. In the offline recognition, the
writing is usually captured optically by a scanner and the completed writing is
available as an image. But, in the online system the two dimensional coordinates
of successive points are represented as function of time and the order of
strokes made by the writer are also available. The online methods have been
shown to be superior to their offline counterparts in recognizing handwritten
characters due to the temporal information available with the former. However,
in the offline systems, the neural networks have been successfully used to yield
comparably high recognition accuracy levels. Several applications including mail
sorting, bank processing, document reading and postal address recognition
require offline handwriting recognition systems. As a result, the offline
handwriting recognition continues to be an active area for research towards
exploring the newer techniques that would improve recognition accuracy.
\par The first important step in any handwritten recognition system is
pre-processing followed by segmentation and feature extraction. Pre-processing
includes the steps that are required to shape the input image into a form
suitable for segmentation. In segmentation, the input image is segmented into
individual characters and then, each character is resized into m * n pixels
towards the training network.
\par The selection of appropriate feature extraction method is probably the
single most important factor in achieving high recognition performance. An
artifial neural network as the backend is used for performing classification and
recognition tasks. In the offline recognition system, neural networks have
emerged as the fast and reliable tools for classification towards achieving high
recognition accuracy.
\par Handwriting recognition is one of the many branches of computer vision, a
vast field involving the mimicry of human capabilities of senses using massive
technology advancements in the field of machine learning and aritificial
intelligence. It mainly comprises of achieving machine vision to automate
human-like tasks. Handwriting recognition has been an age old part of computer
vision and with the advent of the technological boom in deep learning
techniques, systems for this purpose have been successfully developed with
astounding accuracy. Despite this, the offline handwriting recognition problem
belonging to a broad category of recognition still remains unexplored and poses
high research scope.
\end{document}
