\documentclass[main.tex]{subfiles}
\begin{document}

\begin{flushleft}
\Large{\bf{Abstract}}
\end{flushleft}
\vspace{1.5mm}
\justify
\normalsize{Character recognition (CR) has been extensively studied in the last
half century and progressed to a level, sufficient to produce technology driven
applications. Now, the rapidly growing computational power enables the
implementation of the present CR methodologies and also creates an increasing
demand in many emerging application domains, which require more advanced
methodologies. As a subset of CR, special attention is given to the offline
handwriting recognition, since this area requires more research to reach the
ultimate goal of machine simulation of human reading. As a result, this project
aims to create an offline handwritten character recognition model. A
standardized process of achieving offline handwriting recognition which broadly
consists of five steps, namely pre-processing, segmentation, representation,
training and recognition and post processing. Pre-processing can be segmented 
into subtasks which consist of noise removal, binarization i.e. converting the 
image to grayscale, thinning, edge detection, slant estimation and correction, 
skew detection and resizing. Segmentation and representation will consist of 
dividing the cleaned image for feature extraction and then arranging the
divisions into a standard format. Training and recognition will consist of the
deep learning neural network architectural model which will involve a 
combination of convolutional and recurrent neural network layers. Post
processing will consist of stitching the context of the information together to
obtain meaningful text. Finally, a simplistic human interface will act as a 
interactive program enabling user to employ the proposed offline handwriting
recognition application to digitize document images.
}
\end{document}
