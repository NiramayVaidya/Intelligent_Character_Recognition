\documentclass[main.tex]{subfiles}
\begin{document}\

\newpage
\begin{flushleft}
\Large{\bf{Conclusion}}
\end{flushleft}
\vspace{1.5mm}
The overall methodology of performing the task of offline handwriting 
recognition and developing a system prototype for it has been finalized by
referring multiple researh papers which broadly delineate the common steps of
pre-processing, segmentation, representation, training and recognition, and post
processing. Various subtasks as part of these major tasks were studied and
understood as mentioned by the referenced papers and a comparative study was
performed to narrow down on a particular set of subtasks to be executed under
these main tasks. This study is given by-
\begin{center}
\begin{longtable}{ | m{7em} | m{8em} | m{8em} | m{8em} | m{8em} | }
\hline
& \textbf{Paper 1} & \textbf{Paper 2} & \textbf{Paper 3} & \textbf{Paper 4} \\
\hline
\textbf{Pre-processing} & \footnotesize noise removal, binarization, 
	edge detection, dilation and filling & \footnotesize noise reduction 
	(filtering, morphological operations, noise modelling), normalization (skew,
	slant, size normalization, baseline extraction, countour smoothing), 
	compression (thresholding, thinning) & \footnotesize thresholding, noise
	removal & \footnotesize image to matrix tranformation, conversion to
	monochromatic scheme\\
\hline
\textbf{Segmentation} & \footnotesize decomposition into sub-images and distinct
	character labelling & \footnotesize external, internal (explicit, 
	implicit, mixed strategy) & \footnotesize line, word and character
	segmentation & \footnotesize word boundary detection, detection of join
	points between characters of a word \\
\hline
\textbf{Representation} & \footnotesize - & \footnotesize global transformation 
	and series expansion (fourier transforms, gabor transform, wavelets, moments
	and Karhunen-Loeve expansion), statistical representation (zoning, crossings
	, distances, projections), geometrical and topological representation 
	(extraction of counting topological structures, measuring and approximating
	the geometrical properties, coding, graphs and trees) & \footnotesize - &
	\footnotesize - \\
\hline
\textbf{Training and Recognition} & \footnotesize feature extraction using feed 
	forward back propagation neural network & \footnotesize template matching 
	(direct matching deformable templates, elastic and relaxation matching), 
	statistical techniques (parametric and non-parametric recognition, 
	clustering analysis, hidden markov modeling, fuzzy set reasoning), 
	structural techniques (grammatical and graphical methods), neural networks &
	\footnotesize hidden markov model, limited size dynamic lexicon, dynamic 
	programming & \footnotesize using combination of four neural networks with 
	different training hyperparameters \\
\hline
\textbf{Post processing} & \footnotesize - & \footnotesize contextual 
	information retainment using a feedback loop technique & \footnotesize - &
	\footnotesize employing a genetic algorithm to determine the best output out
	of the four distinct outputs generated by the four corresponding neural
	networks \\
\hline
\end{longtable}
\end{center}
\end{document}
